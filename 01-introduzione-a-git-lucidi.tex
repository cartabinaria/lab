\documentclass{beamer}

\usepackage[italian]{babel}
\usepackage[utf8]{inputenc}
\usepackage{graphicx}

\title{Introduzione a \texttt{git}}
\author{Luca Tagliavini, Stefano Volpe}
\institute{Università di Bologna, corso di Laurea in Informatica}
\date{8 novembre 2022}
\logo{\includegraphics[width=0.05\textwidth]{assets/by-nc-sa-4-0.png}}

\AtBeginSection[]{
  \begin{frame}
    \frametitle{In questa sezione}
    \setcounter{tocdepth}{2}
    \tableofcontents[currentsection]
  \end{frame}
}

\begin{document}

\begin{frame} 
  \titlepage
\end{frame}

\section{Controllo di versione}

\subsection{Problemi}
\begin{frame}{Problemi}
\end{frame}

\subsection{Rimedi}
\begin{frame}
  \frametitle{Rimedi}
\end{frame}

\subsection{Ambiente di lavoro}
\begin{frame}
  \frametitle{Ambiente di lavoro}
\end{frame}

\section{Basi di \texttt{git}}

\subsection{\texttt{git init}}
\begin{frame}
  \frametitle{\texttt{git init}}
\end{frame}

\subsection{\texttt{git add}}
\begin{frame}
  \frametitle{\texttt{git add}}
\end{frame}

\subsection{\texttt{git status, commit}}
\begin{frame}
  \frametitle{\texttt{git status, commit}}
\end{frame}

\subsection{\texttt{git log}}
\begin{frame}
  \frametitle{\texttt{git log}}
\end{frame}

\subsection{\texttt{git branch, checkout, switch}}
\begin{frame}
  \frametitle{\texttt{git branch, checkout, switch}}
\end{frame}

\section{Collaborazione remota}

\subsection{\texttt{git remote}}
\begin{frame}
  \frametitle{\texttt{git remote [show]}}
  Una repository locale pu\`o essere collegata ad una remota per facilitare la
  collaborazione, per la distribuzione del codice o per ragioni di backup.
  Per gestire gli endpoint remoti a cui si vuole essere collegati si usano i
  sottocomandi della famiglia \texttt{git remote}. \\
  Per ottenere una lista di remoti collegati alla nostra repository e delle
  loro propriet\`a si ha il comando:
  \begin{semiverbatim}
  \$ git remote [-v]
  \end{semiverbatim}
  Alternativamente si possono ottenere ancora pi\`u informazioni con il comando:
  \begin{semiverbatim}
  \$ git remote show [name]
  \end{semiverbatim}
\end{frame}

\begin{frame}
  \frametitle{\texttt{git remote add, remove}}
  Per configurare un nuovo remoto per la nostra repository possiamo usare il
  comando:
  \begin{semiverbatim}
  \$ git remote add <name> <uri>
  \end{semiverbatim}
  passando un opportuno \texttt{URI}. L'indirizzo pu\`o essere di tipo
  \texttt{HTTP}, \texttt{SSH} ma anche un percorso sul proprio filesystem
  contenente una opportuna repository. Il nome standard per il remoto principale
  \`e \emph{origin}. Nella maggior parte dei provider moderni l'utilizzo di
  \texttt{HTTP(S)} \`e stato deprecato per ragioni di sicurezza. In questa guida
  useremo sempre remoti con protocollo \texttt{SSH} e voi dovreste fare
  altrettando. \\
  \texttt{remove} \`e il sottocomando speculare ad \texttt{add} che consente di
  rimuovere un remoto dato il nome:
  \begin{semiverbatim}
  \$ git remote remove <name>
  \end{semiverbatim}
\end{frame}

\subsection{\texttt{git push}}
\begin{frame}
  \frametitle{\texttt{git push}}
  Una volta aggiunto un endpoint \`e possibile caricare i propri commit alla
  repository remota:
  \begin{semiverbatim}
  \$ git push [-u] [<name>]
  \end{semiverbatim}
  Dover specificare il nome del remoto ogni volta pu\`o diventare tedioso, di
  conseguenza \`e tipico configurare un remoto predefinito, denominato
  \emph{upstream}, eseguendo un \texttt{push} con la flag \texttt{-u}. \\
  % TODO: mhhh
  Nel tempo si \`e esteso il significato del termine \emph{upstream} per
  indicare la repository "originale" nel contesto fork tra progetti.
\end{frame}

\subsection{\texttt{git fetch, pull}}
\begin{frame}
  \frametitle{\texttt{git fetch, pull}}
\end{frame}

\subsection{\texttt{git merge}}
\begin{frame}
  \frametitle{\texttt{git merge}}
\end{frame}

\end{document}
