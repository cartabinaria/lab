\section{Uniamo Scripring a Vim}
\begin{frame}[fragile]{Creiamo un nuovo script}
  Apriamo un nuovo file .sh con vim:
  \begin{figure}
    \begin{lstlisting}
[mattia@pirandello prove]$ vim newScript.sh
\end{lstlisting}
  \end{figure}
  Iniziamo a scrivere entrando dalla Normal Mode alla Insert Mode con "i"

  In prima riga mettiamo come sempre: 
  \begin{figure}
    \begin{lstlisting}[backgroundcolor=\color{expowhite!50!white}]
#!/bin/bash
\end{lstlisting}
  \end{figure}
\end{frame}

\begin{frame}[fragile]{Variabili d'ambiente e comandi}
  Proviamo intanto a vedere se i comandi e le variabili d'ambiente che abbiamo usato funzionano: 
  \begin{figure}
    \begin{lstlisting}[backgroundcolor=\color{expowhite!50!white}]
#!/bin/bash

TIMESTART=$(date)

echo "Script started by ${USER} at ${TIMESTART}"
\end{lstlisting}
  \end{figure}

  Salviamo il file (\texttt{:wq})
  
  Rendiamo il file eseguibile (\texttt{chmod +x newScript.sh})
  
  E runniamolo (\texttt{./newScript.sh})
\end{frame}

\begin{frame}[fragile]{If - else}
  Proviamo ad aggiungere una condizione. Le condizione sono valutate in due modi:
  \begin{itemize}
    \item \texttt{[]} esegue la condizione contenuta
    \item \texttt{[[]]} supporta operatori come \texttt{\&\&}, \texttt{||} 
  \end{itemize}

  Li usiamo per if - then - elif - else:

  \begin{itemize}
    \item if [condizione] then ...
    \item else ...
  \end{itemize}
  Proviamo ad usarli nel file di prima.

  \begin{figure}
    \begin{lstlisting}
[mattia@pirandello prove]$ vim newScript.sh
    \end{lstlisting}
  \end{figure}

  Con G andiamo in fondo al file, e con O scriviamo dalla riga seguente
\end{frame}

\begin{frame}[fragile]{If - else}
  \begin{figure}
    \begin{lstlisting}[backgroundcolor=\color{expowhite!50!white}]
#!/bin/bash
TIMESTART=$(date)
echo "Script started by ${USER} at ${TIMESTART}"
MESSAGE=$({ls /pollegsasso} 2>&1 )
if [[ -n "${MESSAGE}" ]]; then
    echo "Warning: Errors were detected during script execution."
else
    echo "Script completed successfully with no errors."
fi
\end{lstlisting}
  \end{figure}
\end{frame}