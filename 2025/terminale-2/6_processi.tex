\section{Processi}
\begin{frame}{htop}
  Linux è sistema che permette l'esecuzione di più processi simultaneamente. 
  \medskip

  Per vedere i processi in esecuzione (e altre informazioni) è molto comodo il 
  comando \texttt{top} o \texttt{htop} (la versione nuova). \medskip
  
  Se si vuole invece aver uno snapshot (istantanea) dei processi in esecuzione si può 
  utilizzare il comando \texttt{ps}. \medskip

  Per vedere tutti i processi su \texttt{ps} la flag è \texttt{-e}.
\end{frame}

\begin{frame}[fragile]{Pid e kill}
  Per identificare in modo univoco un processo si utilizza un intero positivo
  chiamato \textbf{pid}: Process IDentifier. Il pid può essere ottenuto attraverso il comando \texttt{ps}.\medskip
  
  Se per caso volessimo terminare un processo lo si può fare attraverso il 
  comando \texttt{kill} passando il pid del processo da uccidere. \medskip

  \begin{figure}
    \begin{lstlisting}
[mattia@pirandello prove]$ ps
    PID TTY          TIME CMD
   8239 pts/3    00:00:03 zsh
  20813 pts/3    00:00:00 ps
[mattia@pirandello prove]$ kill 8239
\end{lstlisting}
    \caption{Esempio di \texttt{kill}}
  \end{figure}
\end{frame}