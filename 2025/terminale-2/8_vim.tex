\section{Una veloce introduzione a vim}
\begin{frame}[fragile]{Affrontiamo Vim}
  La scorsa volta abbiamo visto che su Linux esistono due principali editor di testo: nano e vim. \smallskip

  Cerchiamo di capire perché proprio vim, qual è la sua \textbf{filosofia}? 
  \begin{itemize}
    \item Quando si programma la maggior parte del tempo è passato a 
      \textit{modificare} il codice, non a scriverlo.

    \item Modificare il codice include molto altro oltre a scrivere: eliminare,
      sostituire, riordinare, duplicare, formattare, ecc.

    \item Ha senso facilitare tutta la parte di modifica del codice, più che 
      di scrittura effettiva.

    \item Il mouse è una perdita di tempo quando si deve scrivere, se si può
      fare tutto da tastiera in modo efficiente è meglio.

    \item Un'operazione usa il minor numero di tasti possibili.
  \end{itemize}
\end{frame}

\begin{frame}{Modalità}
  In un editor classico ci si trova sempre nella modalità di \textit{modifica},
  ovvero se si schiacciano delle lettere vengono inserite nel file che si sta 
  modificando. \medskip  

  Per la maggior parte degli editor (compreso nano) questa è l'unica modalità 
  presente. \medskip 

  Vim introduce tre modalità principali:
  \begin{itemize}
    \item Normal mode
    \item Insert mode
    \item Visual mode
  \end{itemize}
\end{frame}

\begin{frame}{Modalità}
  Vim introduce tre modalità principali:
  \begin{itemize}[<+->]
    \item \textbf{Normal mode}: Modalità predefinita di vim. La si attiva con 
      il tasto \texttt{ESC}. Permette di muoversi all'interno del documento in 
      modo rapido. Di eliminare, copiare, incollare e spostare parti di testo.

    \item \textbf{Insert mode}: Modalità classica per l'inserimento e la 
      scrittura di testo. È la stessa che si trova nella maggior parte degli 
      editor. Ci si accede principalmente usando il tasto '\texttt{i}', ma anche '\texttt{I}', 
      '\texttt{a}', '\texttt{A}', ecc.

    \item \textbf{Visual mode}: Permette di selezionare parti di testo, sia 
      verticalmente che orizzontalmente. Ci si accede usando il tasto 'v'.
  \end{itemize}
\end{frame}

\begin{frame}{Normal mode}
  La \textbf{normal mode} è la predefinita e quella su cui si dovrebbe sempre 
  stare se non si necessitano di altre modalità. Per entrarci usare il tasto
  \texttt{ESC}. \medskip

  Per spostarsi all'interno del testo in normal mode si usano le seguenti 
  lettere e NON le freccette:
  \begin{itemize}
    \item \texttt{j}: Per spostarsi verso il basso di una riga
    \item \texttt{k}: Per spostarsi verso l'alto di una riga
    \item \texttt{h}: Per spostarsi a sinistra di un carattere
    \item \texttt{l}: Per spostarsi a destra di un carattere
  \end{itemize}
  Il motivo per cui si utilizzano queste lettere e non le freccette è il fatto 
  che lo \textit{spostamento della mano} per raggiungere le freccette è troppo grande
  e richiederebbe troppo tempo per la filosofia di vim.
\end{frame}

\begin{frame}{Normal mode}
  Altre combinazioni di tasti base in normal mode:
  \begin{itemize}
    \item \texttt{w}: si sposta all'inizio della parola successiva
    \item \texttt{e}: si sposta alla fine della parola successiva
    \item \texttt{b}: si sposta all'inizio della parola precedente
    \item \texttt{gg}: si sposta all'inizio del file
    \item \texttt{G}: si sposta alla fine del file
    \item \texttt{dd}: cancella la riga su cui si trova il cursore
    \item \texttt{yy}: copia la riga su cui si trova il cursore
    \item \texttt{p}: incolla il testo copiato nella riga successiva al cursore
    \item \texttt{x}: elimina il carattere sotto il cursore
    \item \texttt{u}: annulla l'ultima operazione fatta
    \item \texttt{ctrl + r}: ripete l'ultima operazione annullata (il contrario di '\texttt{u}')
  \end{itemize}
\end{frame}

\begin{frame}{Insert mode}
  Si utilizza la \textbf{insert mode} quando si vuole scrivere del testo. Per
  entrarci dalla normal mode si possono usare i seguenti tasti:
  \begin{itemize}
    \item \texttt{i}: si inizia a digitare prima della lettera sotto il cursore
    \item \texttt{a}: si inizia a digitare dopo della lettera sotto il cursore
    \item \texttt{I}: si inizia a digitare all'inizio della riga
    \item \texttt{A}: si inizia a digitare alla fine della riga
    \item \texttt{o}: si inizia a digitare in una nuova linea sottostante
    \item \texttt{O}: si inizia a digitare in una nuova linea sovrastante
    \item \texttt{cc}: si cancella la riga su cui si trova il cursore e si inizia a 
      scrivere al suo posto
  \end{itemize}
\end{frame}

\begin{frame}{How to exit vim}
  Uscire da vim è una delle operazioni più complesse. \medskip 

  Molti dicono che l'unico modo sia riavviare il pc. \medskip 

  In realtà basta sapere come funziona! \medskip

  Per uscire da vim bisogna trovarsi in \textbf{normal mode} e digitare \textbf{\texttt{:q}} e 
  successivamente premere \texttt{invio}.\medskip

  Se il file è stato modificato dall'apertura il comando non funzionerà perché
  vim non sa se vuoi salvare il file o no. Per salvarlo (ed uscire) si usa 
  \textbf{\texttt{:wq}} e per non salvarlo (ed uscire) \textbf{\texttt{:q!}} \medskip

  Per salvare quindi un file senza uscire basta \textbf{\texttt{:w}}
\end{frame}

