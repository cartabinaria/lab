\section{Sostituzione di comandi}
\begin{frame}[fragile]{La pipe non riesce a fare tutto}
  Con la pipe \textbf{|} possiamo combinare più comandi assieme, però non riusciamo a fare
  ancora tutto. \textit{Se volessimo vedere il tipo di tutti i file che hanno la parola zip nel nome
  dentro la cartella \texttt{/bin} come facciamo?}
  \begin{figure}
    \begin{lstlisting}
[mattia@pirandello prove]$ ls -d /bin/* | grep zip | file
Usage: file [...]
\end{lstlisting}
    \caption{Esempio \textbf{SBAGLIATO}}
  \end{figure}
  
  Questo comando restituisce un errore perché il comando \texttt{file} non 
  riesce ad interpretare lo standard input tramite la pipe, infatti \texttt{file} vuole gli input come argomenti: \texttt{file ...}
\end{frame}

\begin{frame}[fragile]{Sostituzione di comandi}
  La shell ammette una sintassi per la sostituzioni di comandi all'interno di
  un'espressione. \medskip

  Sintassi: \texttt{\$(comando)} \medskip

  La shell sostituirà l'intera espressione al risultato del comando:
  \begin{figure}
    \begin{lstlisting}
[mattia@pirandello prove]$ echo $(ls)
1  capybara  castoro  criceto  file  file1  file2  mangusta
\end{lstlisting}
    \caption{Esempio di sostituzione del comando \texttt{ls}}
  \end{figure}
\end{frame}

\begin{frame}[fragile]{Sostituzione di comandi}
  Ritornando a quello che volevamo fare prima: \textit{Se volessimo vedere il 
  tipo di tutti i file che hanno la parola zip nel nome dentro la cartella 
  \texttt{/bin} come facciamo?} \textbf{Con la sostituzione di comandi!}

  \begin{figure}
    \begin{lstlisting}
[mattia@pirandello prove]$ file $(ls -d /bin/* | grep zip)
\end{lstlisting}
    \caption{Esempio di più comandi}
  \end{figure}
\end{frame}

\begin{frame}[fragile]{Matematica base}
  Si possono fare anche operazioni matematiche con questo tipo di sostituzione:
  \begin{figure}
    \begin{lstlisting}
[mattia@pirandello prove]$ echo $((1 + 1))
2
[mattia@pirandello prove]$ echo $((9 * 5))
45
[mattia@pirandello prove]$ echo 3GB are $((3 * 1024))MB
3GB are 3072MB
\end{lstlisting}
    \caption{Esempio di più comandi}
 \vspace{0.3em}
    \footnotesize\textbf{Nota:} è necessario usare la doppia parentesi \texttt{(())} in quanto vogliamo che sia un'espressione aritmetica.
  \end{figure}
\end{frame}

