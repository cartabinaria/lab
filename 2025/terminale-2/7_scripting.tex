\section{Bash scripting}
\begin{frame}[fragile]{Quante shell!}
  Esistono varie shell:
  \begin{itemize}
    \item \texttt{sh} (Bourne Shell)
    \item \texttt{bash} (Bourne Again Shell)
    \item \texttt{zsh} (Z sh, da Zhong Shao)
    \item \texttt{fish} (Friendly Interactive sh)
    \item \texttt{csh}
    \item \texttt{tcsh}
    \item \texttt{dsh}
    \item \texttt{ksh} ...
  \end{itemize}
  Tutte servono per interagire con il sistema operativo, però sono tutte 
  leggermente diverse sotto alcuni aspetti.\medskip

  Di solito la predefinita nei sistemi Linux è \texttt{bash}.
\end{frame}

\begin{frame}[fragile]{Bash scripting}
  Spesso si vogliono automatizzare dei comandi della shell, oppure eseguirne più
  in "una volta sola". \medskip

  La shell (e in particolare bash) permette di scrivere degli script. \medskip

  Uno \textbf{script} è un file di testo con un elenco di comandi che vengono eseguiti in
  sequenza da bash (dall'interprete che specificate)
  .\medskip

  Per specificare che interprete da usare è necessario indicare nella prima riga
  del file il percorso assoluto all'eseguibile della shell: \smallskip

  \begin{figure}
    \begin{lstlisting}[backgroundcolor=\color{expowhite!50!white}]
#!/bin/bash
\end{lstlisting}
    \caption{Riga per indicare bash come interprete}
  \end{figure}
\end{frame}



\begin{frame}[fragile]{Creare un nuovo script}
  Per eseguire uno script bash serve un file (lo script). Per convenzione gli script bash si salvano con estensione \textbf{.sh}
  \medskip


  \begin{figure}
    \begin{lstlisting}
[mattia@pirandello prove]$ nano script.sh
\end{lstlisting}
  \end{figure}\medskip

  \begin{figure}
    \begin{lstlisting}[backgroundcolor=\color{expowhite!50!white}]
#!/bin/bash
echo "Reading programs"
PROG=$(ls -d /bin/*)
NUM=$(echo "$PROG" | grep "ca" | wc -l)
echo "Numero di programmi con 'ca' è $NUM"
\end{lstlisting}
    \caption{Cosa fa?}
  \end{figure}
\end{frame}


\begin{frame}[fragile]{Esempio di script}
  Questo script fa le seguenti operazioni:
  \begin{enumerate}
    \item Stampa sullo \textit{standard output} "\textit{Reading programs}"
    \item Salva tutti i programmi presenti in \texttt{/bin} nella variabile 
      \texttt{PROG}.
    \item Salva nella variabile \texttt{NUM} il numero di tutte le stringhe 
      presenti nella variabile \texttt{PROG} che hanno al loro interno "ca".
    \item Stampa il valore presente nella variabile \texttt{NUM}.
  \end{enumerate}
\end{frame}

\begin{frame}[fragile]{Eseguire uno script}
  Come abbiamo visto la scorsa volta, bisogna rendere lo script eseguibile aggiungendogli il permesso: \code{chmod +x script.sh}
  \begin{figure}
    \begin{lstlisting}
[mattia@pirandello prove]$ ls -al
-rw-r--r-- 1 mattia mattia Nov 06 18:00 script.sh
[mattia@pirandello prove]$ chmod +x script.sh
-rwxr-xr-x 1 mattia mattia Nov 06 18:00 script.sh
\end{lstlisting}
  \end{figure}

  Per eseguirlo basta aggiungere \texttt{./} prima del nome:
  \begin{figure}
    \begin{lstlisting}
[mattia@pirandello prove]$ ./script.sh
\end{lstlisting}
    \caption{Esecuzione di uno script bash}
  \end{figure}
\end{frame}