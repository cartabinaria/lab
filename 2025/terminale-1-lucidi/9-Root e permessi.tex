\section{Root e permessi}

\begin{frame}{Sistema di permessi}
  Linux è un sistema multi-utente, quindi più utenti posso usare simultaneamente la stessa macchina. \bigskip

  Per questo è necessario avere un sistema di permessi 
  adeguato.\bigskip

  Il sistema di permessi utilizzato da Linux non è semplice, e di seguito sarà data soltanto un'introduzione.\bigskip
\end{frame}

\begin{frame}{Root}
  In tutti i sistemi Linux esiste un unico utente che ha i permessi per seguire
  qualsiasi operazione: \textbf{root}\bigskip

  È l'amministratore del sistema.\bigskip

  NON deve essere MAI usato come utente se non per le operazioni strettamente
  necessarie.\bigskip

  Per aprire una \textit{shell} come utente \texttt{root} è possibile digitare
  il comando: \texttt{su}\bigskip

  Il comando chiederà quindi la password di \texttt{root} (che generalmente è
  impostata durante l'installazione del sistema) e se corretta aprirà una 
  \textit{shell} con i privilegi di amministratore.
\end{frame}

\begin{frame}{Permessi su un file}
  Torniamo ad usare il nostro utente (se siamo root facciamo \texttt{exit}). 
  Dentro la cartella \texttt{/etc} esiste un file chiamato \texttt{shadow}. 
  Se proviamo a leggerlo con \texttt{less} otteniamo: \textbf{/etc/shadow: 
  Permission denied}\bigskip

  Questo significa che il nostro utente non ha i permessi per leggere il file.
  Ma come potevamo saperlo a priori senza tentare di leggerlo?\bigskip

  Una \textit{flag} molto usata per il comando \texttt{ls} è 
  \texttt{-l}. (o \texttt{-al}) che permettono di vedere molte più informazioni 
  sui file presenti in una directory.
\end{frame}

\begin{frame}{Esempio di ls -al}
  \begin{figure}
    \begin{tikzpicture}
      \node[inner xsep=0pt,brown] (a) {\texttt{d}};

      \node[inner xsep=0pt,right=0pt of a, blue] (b) {\texttt{rwxr-xr-x}};
      \node[inner xsep=0pt,right=5pt of b, green!50!black] (c) {\texttt{matti}};
      \node[inner xsep=0pt,right=5pt of c, orange] (d) {\texttt{matti}};
      \node[inner xsep=0pt,right=5pt of d] (e) 
      {\texttt{4096 Nov  4 17:00 castoro}};

      \draw [brown] (a.south west) -- (a.south east);
      \draw [blue] (b.north west) -- (b.north east);
      \draw [green!50!black] (c.south west) -- (c.south east);
      \draw [orange] (d.north west) -- (d.north east);

      \coordinate (a1) at (a.south);
      \coordinate (b1) at (b.north);
      \coordinate (c1) at (c.south);
      \coordinate (d1) at (d.north);

      \draw[brown] ($(a1)+(-90:0.1)$)--($(a1)+(-90:1)$) 
      node[fill=white,font=\normalsize\sffamily,inner sep=2pt] 
      {Directory o file};

      \draw[blue] ($(b1)+(90:0.1)$)--($(b1)+(90:1.5)$) 
      node[fill=white,font=\normalsize\sffamily,inner sep=5pt, text 
      width=3.5cm] {Permessi di lettura, scrittura ed esecuzione};

      \draw[green!50!black] ($(c1)+(-90:0.1)$)--($(c1)+(-60:2)$) 
      node[fill=white,font=\normalsize\sffamily,inner sep=2pt] 
      {User owner};

      \draw[orange] ($(d1)+(90:0.1)$)--($(d1)+(50:2)$) 
      node[fill=white,font=\normalsize\sffamily,inner sep=5pt,] 
      {Group owner};
    \end{tikzpicture}
  \end{figure}
\end{frame}

\begin{frame}{Read, write, execute}
  Ogni file ha un stringa formata da \textbf{9 bit} che determina quali 
  permessi specifici ha quel file rispetto all'\textit{utente}, il 
  \textit{gruppo} e gli \textit{altri}.\bigskip

  I 9 bit sono suddivisi in \textbf{gruppi di 3}: il primo è specifico per 
  l'\textit{utente}, il secondo è specifico per il \textit{gruppo} e i rimanenti
  sono per tutti gli \textit{altri}.
  \begin{figure}
    \begin{tikzpicture}
      \node[inner xsep=0pt,orange] (a) {\texttt{rwx}};

      \node[inner xsep=0pt,right=10pt of a, blue] (b) {\texttt{rwx}};
      \node[inner xsep=0pt,right=10pt of b, green!50!black] (c) {\texttt{rwx}};

      \draw [brown] (a.south west) -- (a.south east);
      \draw [blue] (b.north west) -- (b.north east);
      \draw [green!50!black] (c.south west) -- (c.south east);

      \coordinate (a1) at (a.south);
      \coordinate (b1) at (b.north);
      \coordinate (c1) at (c.south);

      \draw[brown] ($(a1)+(-90:0.1)$)--($(a1)+(-90:1)$) 
      node[fill=white,font=\normalsize\sffamily,inner sep=2pt] 
      {User};

      \draw[blue] ($(b1)+(90:0.1)$)--($(b1)+(90:1)$) 
      node[fill=white,font=\normalsize\sffamily,inner sep=5pt] {Group};

      \draw[green!50!black] ($(c1)+(-90:0.1)$)--($(c1)+(-90:1)$) 
      node[fill=white,font=\normalsize\sffamily,inner sep=2pt] 
      {Other};
    \end{tikzpicture}
  \end{figure}
\end{frame}

\begin{frame}{Tabella con i permessi}
  \begin{table}
    \begin{tabularx}{\textwidth}{|l|l|X|}
      \hline
      & File  & Directory  \\ \hline
      r & Permette la lettura di un file & Permette di vedere il contenuto se 
      anche \texttt{x} è segnato\\ \hline
      w & Permette di scrivere sul file  & Permette di creare ed eliminare file 
      dentro la directory se \texttt{x} è segnato \\ \hline
      x & Permette di eseguire un file   & Permette di entrare nella 
      directory \\ \hline
    \end{tabularx}
    \caption{Significato dei permessi per file e directory}
  \end{table}
\end{frame}


\begin{frame}{Cambiare permessi wrx}
  Per cambiare i permessi lettura, scrittura ed esecuzione si utilizza il 
  comando \texttt{chmod}.\bigskip

  Sintassi: \texttt{chmod [PART][ACTION][PERMISSION] file}\bigskip

  \only<1> {
    Al posto di \texttt{[PART]} è necessario specificare la 
    parte che si vuole modificare:
    \begin{itemize}
      \item user: si utilizza \texttt{u}
      \item group: si utilizza \texttt{g}
      \item others: si utilizza \texttt{o}
      \item all: si utilizza \texttt{a}
    \end{itemize}
  }

  \only<2> {
    Al posto di \texttt{[ACTION]} è necessario specificare la 
    l'azione da compiere
    \begin{itemize}
      \item \texttt{+}: Aggiunge il permesso
      \item \texttt{-}: Rimuove il permesso
      \item \texttt{=}: Assegna esattamente quel permesso
    \end{itemize}
  }

  \only<3> {
    Al posto di \texttt{[PERMISSION]} è necessario specificare la 
    il permesso o i permessi da moficare:
    \begin{itemize}
      \item \texttt{r}: Read
      \item \texttt{w}: Write
      \item \texttt{x}: Execute
    \end{itemize}
  }

  \bigskip Per eseguire il comando sono necessari i privilegi di root.
\end{frame}

\begin{frame}{Esempi \texttt{chmod}}
  \begin{itemize}
    \item \texttt{chmod u+x pippo} Rende pippo un file eseguibile per l'utente
    \item \texttt{chmod o-w pippo} Rimuove la possibilità a tutti gli utenti
      diversi dall'owner del file e non appartenenti al gruppo del file di 
      scrivere su pippo.
    \item \texttt{chmod g+r pippo} Rende pippo leggibile al gruppo
    \item \texttt{chmod g+x pippo} Rende pippo eseguibile dal gruppo
    \item \texttt{chmod u=rwx,g=,o= pippo} Rende pippo leggibile, scrivibile ed
      eseguibile per l'utente. Inoltre rimuove tutti i permessi dal gruppo e 
      altri.
  \end{itemize}
\end{frame}

\begin{frame}{Cambiare owner e group}
  Per cambiare l'owner di un file e il gruppo possiamo usare il comando 
  \texttt{chown}.\bigskip

  Sintassi: \texttt{chown user:group file}\bigskip

  Per eseguire il comando sono necessari i privilegi di root.
\end{frame}
