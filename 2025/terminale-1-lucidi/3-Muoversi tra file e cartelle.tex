\section{Muoversi tra i file e le cartelle}

\subsection{Elencare i file - ls}
\begin{frame}[fragile]{Elencare i file}
  Le operazioni più importanti sono legate alla gestione dei file. Proviamo a 
  digitare il comando \texttt{ls} nella shell.\bigskip

    \begin{lstlisting}
[matti@pirandello ~]$ ls
Desktop Documents Downloads Music Pictures 
Public  Templates Video
[matti@pirandello ~]$
    \end{lstlisting}

  Sono gli stessi che vediamo nell'interfaccia
\end{frame}

\subsection{Muoversi tra le cartelle - cd}
\begin{frame}[fragile]{Entrare in una cartella}
  Per muoverci tra le cartelle con la \texttt{shell}
  possiamo usare il comando \texttt{cd}.\bigskip

  Per entrare in una cartella è necessario indicare in quale ci vogliamo muovere 
  visto che potrebbe essercene più di una.\bigskip

  Per farlo è quindi necessario fornire al comando \texttt{cd} un 
  \textbf{argomento}.\bigskip

    \begin{lstlisting}
[matti@pirandello ~]$ cd Documents/
[matti@pirandello Documents]$
    \end{lstlisting}
\end{frame}

\begin{frame}[fragile]{Uscire da una cartella}
  Per uscire da una cartella il comando è \texttt{cd ..}\bigskip
    \begin{lstlisting}
[matti@pirandello Documents]$ cd ..
[matti@pirandello ~]$
    \end{lstlisting}

  L'argomento \texttt{..} indica sempre la cartella genitore di quella attuale.
\end{frame}

\begin{frame}{Current working directory}
  In Linux le cartelle si chiamano \textit{directory}. Per stampare il 
  \textit{path} della cartella corrente usiamo il comando: \texttt{pwd}.\bigskip

  Il path assume la seguente forma: \texttt{/home/matti}.\bigskip

  Il carattere \texttt{/} viene utilizzato come separatore, quindi \texttt{matti}
  è dentro la cartella \texttt{home}. \bigskip 

  Prima di \texttt{home} c'è uno \texttt{/}
  che in Linux indica la radice del \textbf{file-system}. Quindi \texttt{home} a sua
  volta è contenuta in \texttt{/}
\end{frame}