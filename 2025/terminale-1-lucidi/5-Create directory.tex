\section{Creare directory e file}
\begin{frame}[fragile]{Creare directory}
  Per creare una directory esiste il comando \texttt{mkdir}.
  \begin{figure}
    \begin{lstlisting}[basicstyle=\small]
[matti@pirandello ~]$ cd Documents/
[matti@pirandello Documents]$ mkdir prove
[matti@pirandello prove]$ cd prove/
[matti@pirandello prove]$ mkdir castoro 
[matti@pirandello prove]$ ls 
castoro
[matti@pirandello prove]$ cd castoro/
[matti@pirandello castoro]$
    \end{lstlisting}
    \caption{esempi del comando mkdir}
  \end{figure}

  Sintassi: \texttt{mkdir path/to/directory}
\end{frame}

\begin{frame}[fragile]{Creare file}
  Per creare un file esistono molti modi, ma il più semplice è il comando 
  \texttt{touch}.
  \begin{figure}
    \begin{lstlisting}[basicstyle=\small]
[matti@pirandello castoro]$ cd ..
[matti@pirandello prove]$ touch lontra 
[matti@pirandello prove]$ ls 
castoro lontra
[matti@pirandello prove]$
    \end{lstlisting}
    \caption{esempi del comando touch}
  \end{figure}

  Sintassi: \texttt{touch path/to/file} \bigskip

  \footnotesize
  Il comando \texttt{touch} in realtà serve per cambiare la data di modifica di 
  un file, ma se non esiste allora viene creato.\bigskip
\end{frame}
