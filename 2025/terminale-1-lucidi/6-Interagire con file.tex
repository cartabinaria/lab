\section{Interagire con file}

  \subsection{Distinguere i tipi di file - \texttt{file}}
\begin{frame}[fragile]{A chi serve Nautilus?}
  Per sapere di che tipo è un file possiamo usare il comando \texttt{file}:
  \begin{figure}
    \begin{lstlisting}[basicstyle=\footnotesize]
[matti@pirandello prove]$ ls 
castoro lontra
[matti@pirandello prove]$ file castoro 
castoro: directory
[matti@pirandello ~]$ file lontra
lontra: empty
// Se lontra contenesse del testo
// lontra: ASCII text
[mattia@pirandello ~]$
    \end{lstlisting}
    \caption{esempi del comando file}
  \end{figure}
  In linux i file non hanno bisogno di un'estensione, è quindi molto utile 
  questo comando.
\end{frame}

\subsection{Copiare, spostare ed eliminare}
\begin{frame}[fragile]{Copiare file}
  Il comando per copiare dei file è \texttt{cp}. il suo utilizzo è 
  principalmente: \texttt{cp file/da/copiare destinazione}
  \begin{figure}
    \begin{lstlisting}[basicstyle=\small]
[matti@pirandello prove]$ ls 
castoro lontra
[matti@pirandello prove]$ cp lontra criceto 
[matti@pirandello prove]$ ls 
castoro criceto lontra
[matti@pirandello prove]$
    \end{lstlisting}
    \caption{esempi del comando cp}
  \end{figure}
  Sono ammessi sia \textit{path} assoluti sia relativi
\end{frame}

\begin{frame}[fragile]{Spostare file}
  Il comando per spostare (tagliare) dei file è \texttt{mv}. Si usa come il 
  comando di copia: \texttt{mv file/da/muovere 
  destinazione}
  \begin{figure}
    \begin{lstlisting}[basicstyle=\small]
[matti@pirandello prove]$ ls 
castoro criceto lontra
[matti@pirandello prove]$ cd castoro/ 
[matti@pirandello castoro]$ mv ../lontra .
[matti@pirandello castoro]$ ls
lontra
[matti@pirandello castoro]$
    \end{lstlisting}
    \caption{esempi del comando mv}
  \end{figure}
  La destinazione \texttt{.} indica la directory corrente.
\end{frame}

\begin{frame}[fragile]{Rinominare file}
  Il comando \texttt{mv} permette anche di rinominare i file:
  \begin{figure}
    \begin{lstlisting}[basicstyle=\normalsize]
[matti@pirandello castoro]$ touch cubo
[matti@pirandello castoro]$ ls
cubo lontra
[matti@pirandello castoro]$ mv cubo triangolo
[matti@pirandello castoro]$ ls
lontra triangolo
[matti@pirandello castoro]$
    \end{lstlisting}
    \caption{esempi del comando mv}
  \end{figure}
\end{frame}

\begin{frame}{Now I am become death, the destroyer of files}
  Il comando più pericoloso in linux è indubbiamente \texttt{rm}.\bigskip

  \texttt{rm nome/file} elimina il file passato come argomento.\bigskip

  Non si può tornare indietro, una volta eliminato un file è perso per 
  sempre!\bigskip

  \texttt{rm} non funziona come il cestino di Windows con cui puoi ripristinare file,
  la \textit{shell} si aspetta che voi sappiate esattamente quello che state 
  facendo e non si preoccupa se questo può distruggere il sistema.
\end{frame}

\begin{frame}[fragile]{Copiare una cartella}
  Per copiare le directory il comando \texttt{cp} deve funzionare in modalità 
  \textit{ricorsiva}, per permettere la copia di tutti gli elementi all'intero
  della directory.\bigskip

  Per copiare una cartella è quindi necessario aggiungere la \textit{flag} 
  \texttt{-r}
  \begin{figure}
    \begin{lstlisting}[basicstyle=\footnotesize]
[matti@pirandello prove]$ ls 
castoro criceto
[matti@pirandello prove]$ cp castoro capybara 
cp: -r not not specified; omitting directory 'castoro'
castoro criceto
[matti@pirandello prove]$ cp -r castoro capybara 
[matti@pirandello prove]$ ls 
capybara castoro criceto
    \end{lstlisting}
    \caption{esempi del comando cp ricorsivo}
  \end{figure}
\end{frame}

\subsection{Flags - man}
\begin{frame}{Flags}
  Le \textit{flag} sono un modo per estendere le funzionalità di un 
  comando.\bigskip

  Vengono specificate dopo il comando e sono precedute da un trattino 
  \texttt{-}\bigskip

  Per comodità sono di una sola lettera, ma in certi casi possono essere anche 
  più verbose es. \texttt{--recursive}.\bigskip

  Si possono combinare più \textit{flag} concatenando le lettere dopo il 
  trattino: \texttt{-r -t} è equivalente a \texttt{-rt}
\end{frame}

\begin{frame}{Man}
  Esiste un comando per leggere il \textit{manuale} di un comando: 
  \texttt{man}.\bigskip

  Potreste trovarvi in situazioni in cui non saprete come usare un comando,
  non avrete accesso a internet per cercare o non avrete interfacce grafiche...
  li il comando man vi tornerà molto utile.\bigskip

  Sintassi: \texttt{man comando}\bigskip

  Si apre un visualizzatore di testo integrato nel terminale chiamato \texttt{less}.
  Per navigare sono usati i seguenti comandi:
  \begin{columns}[T] % The [T] option aligns the columns' top lines
    \begin{column}{.5\textwidth} % First column
      \begin{itemize}
        \item \texttt{j}: Muoversi verso il basso
        \item \texttt{k}: Muoversi verso l'alto
        \item \texttt{g}: Inizio del file
      \end{itemize}
    \end{column}
    \begin{column}{.5\textwidth} % Second column
      \begin{itemize}
        \item \texttt{/name}: Cerca la stringa \texttt{name}
        \item \texttt{n}: Va all'occorrenza successiva
        \item \texttt{q}: Esce e torna al prompt
      \end{itemize}
    \end{column}
  \end{columns}
\end{frame}

\subsection{Lettura di file - less}
\begin{frame}{Less}
  Esiste un lettore di testo integrato nella \textit{shell} chiamato 
  \texttt{less}. (evoluzione di \texttt{more} presente nei primi s.o. Unix)\bigskip

  \texttt{less} è in realtà un comando che permette di leggere file di 
  testo.\bigskip

  Sintassi: \texttt{less nome/file}\bigskip

  \texttt{less} è molto veloce a leggere file di testo di grandi dimensioni.
\end{frame}

\subsection{Editor di testo}
\begin{frame}[fragile]{Nano e Vim}
  Su Linux esistono due principali editor di testo: nano e vim. \smallskip

  Nano è un editor semplice che ha i comandi scritti a schermo per evitare di
  scordarseli. \smallskip

  La filosofia di vim invece è diversa. Facciamo qualche osservazione:
  \begin{itemize}
    \item Quando si programma la maggior parte del tempo è passato a 
      \textit{modificare} il codice, non a scriverlo

    \item Modificare il codice include molto altro oltre a scrivere: eliminare,
      sostituire, riordinare, duplicare, formattare, ecc.

    \item Ha senso facilitare tutta la parte di modifica del codice, più che 
      di scrittura effettiva

    \item Il mouse è una perdita di tempo quando si deve scrivere, se si può
      fare tutto da tastiera in modo efficiente è meglio

    \item Un'operazione usa il minor numero di tasti possibili.
  \end{itemize}

  Approfondiremo vim più avanti, per il momento usiamo nano.
\end{frame}


\subsection{Scrivere in un file di testo}
\begin{frame}[fragile]{Scrivere in un file di testo}
  Per scrivere in un file di testo esiste il comando \texttt{nano}.\bigskip

  Sintassi: \texttt{nano nome/file}\bigskip

  Una volta dentro l'editor si può scrivere normalmente. Per salvare e uscire
  si usa la combinazione di tasti \texttt{ctrl + x}, poi \texttt{y} per 
  confermare il salvataggio e infine \texttt{invio} per salvare con lo stesso
  nome.\bigskip

  Per uscire senza salvare si usa sempre \texttt{ctrl + x} ma poi si preme
  \texttt{n} per non salvare.
\end{frame}

\begin{frame}[fragile]{Esempio di nano}
  \begin{figure}
    \begin{lstlisting}[basicstyle=\footnotesize]
[matti@pirandello castoro]$ ls
lontra triangolo
[matti@pirandello castoro]$ nano lontra
  \end{lstlisting}
  \end{figure}
  \begin{figure}
    \begin{lstlisting}[basicstyle=\tiny\ttfamily, frame=single, rulecolor=\color{black}]
  GNU nano 4.8                      lontra                      Modified  

  Hello World!
~
~
~
~
~
~
^G Help     ^O Write Out    ^W Where Is     ^K Cut      ^T Execute
^X Exit     ^R Read File    ^\ Replace      ^U Text     ^J Justify
    \end{lstlisting}
    \caption{Scrivere "Hello World!" nel file \texttt{lontra} usando nano.}
  \end{figure}
\end{frame}

\subsection{Operazioni testuali}
\begin{frame}{Contare caratteri, linee, ecc. - wc}
  Esiste un comando per contare i caratteri, le linee e altre informazioni
  all'interno di un file di testo: \texttt{wc}\bigskip

  Sintassi: \texttt{wc nome/file}\bigskip

  Senza nessuna \textit{flag} stampa:
  \begin{itemize}
    \item Il numero di righe
    \item Il numero di parole
    \item Il numero di bytes
  \end{itemize}
  Per stampare il numero di caratteri: \texttt{wc -c nome/file}
  Per stampare il numero di linee: \texttt{wc -l nome/file}
\end{frame}

\begin{frame}[fragile]{Ricerca di una stringa - grep}
  Uno dei comandi più potenti per la ricerca di stringhe in un file è \texttt{grep}\bigskip

  Sintassi: \texttt{grep "string" file}\bigskip

  Esiste la \textit{flag} \texttt{-i} per la ricerca case-insensitive.\bigskip

  \begin{figure}
    \begin{lstlisting}[basicstyle=\footnotesize]
[matti@pirandello castoro]$ grep -i "hello" lontra
Hello World!
[matti@pirandello castoro]$
    \end{lstlisting}
    \caption{Ricerca di una stringa in un file}
  \end{figure}
\end{frame}

\begin{frame}[fragile]{Concatenazione o lettura? - cat}
  Il comando \texttt{cat} è nato per concatenare più file.\bigskip

  Per renderlo però completamente funzionante abbiamo bisogno dell'operatore
  di ridirezione che verrà spiegato più avanti.\bigskip

  Possiamo però usarlo per leggere file generalmente corti. Spesso è più rapido
  da usare di \texttt{less}, anche perchè stampa il file completo sul terminale. \bigskip

  \begin{figure}
    \begin{lstlisting}[basicstyle=\footnotesize]
[matti@pirandello castoro]$ cat lontra
Hello World!
[matti@pirandello castoro]$
    \end{lstlisting}
      \caption{Stampa di un file tramite cat}
  \end{figure}
\end{frame}

\begin{frame}{Testa e coda - head, tail}
  Il comando \texttt{cat} prende uno o più file in input e li stampa sul 
  terminale.\bigskip

  Esiste anche \texttt{head} che stampa solo le prime 10 righe di un file (si
  possono modificare con la \textit{flag} \texttt{-n numero}).\medskip

  Da notare che la \textit{flag} precedente ha preso un parametro.\bigskip

  Esiste anche \texttt{tail} che fa esattamente la stessa cosa di \texttt{head}, 
  ma partendo dalla fine del file.
\end{frame}
