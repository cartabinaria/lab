\subsection{I primi passi in un terminale}
\begin{frame}{Aprire un terminale}
  Per utilizzare una \texttt{shell} è necessario disporre di un emulatore di 
  terminale.\bigskip
  
  Per comunicare con la \texttt{shell} attraverso l'emulatore di terminale
  abbiamo bisogno di usare i \textbf{comandi}.\bigskip
\end{frame}

\subsection{Comandi}
\begin{frame}[fragile]{Comandi}
  Un comando rappresenta una richiesta di eseguire una determinata operazione al
  sistema operativo.\bigskip

  Sintassi: \texttt{comando [opzioni] [argomenti]}

  \begin{figure}
    \begin{lstlisting}
[matti@pirandello ~]$ date
Thu Nov 4 05:00:00 PM CEST 2025
[matti@pirandello ~]$
    \end{lstlisting}
  \end{figure}
  \begin{flushright}
    \footnotesize
      Stampa data, ora e qualche altra informazione come il fuso orario.
  \end{flushright}
\end{frame}
