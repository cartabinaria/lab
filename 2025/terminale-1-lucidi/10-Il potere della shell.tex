\section{Pipe e ridirezione}
\begin{frame}{Il potere della shell}
  Finora abbiamo visto soltanto comandi base e usati singolarmente. Spesso sono
  utili usati cos`i come abbiamo visto, ma diventano molto potenti se usati in
  combinazione con altri comandi.\bigskip

  Come si fa però a combinare più comandi? \bigskip 

  Ricordiamoci che:
  \begin{itemize}
    \item Un comando restituisce sempre qualcosa sullo standard output
    \item Lo standard output è considerato come un file dal sistema
    \item La maggior parte dei comandi visti fino ad adesso hanno la 
      possibilità di prendere in input lo standard output invece che un file 
      classico
  \end{itemize}
\end{frame}

\begin{frame}{Cosa possiamo fare dalla shell}
  \begin{figure}
    \begin{tikzpicture}
      \draw (-1.5,3) rectangle ++(3, 1);
      \draw (-1.5,-1.95) rectangle ++(3, 1);
      \draw [->] (0,3) -- (0, 2);
      \draw [->] (0,0.1) -- (0, -0.9);
      \node [cylinder, shape border rotate=90, draw,minimum height=2cm,minimum width=1.5cm] at (0,1) {};

      \node at (0, 1) {buffer};
      \node at (0, 3.5) {command 1};
      \node at (0, -1.5) {command 2};
      \node at (1, -0.5) {output};
      \node at (1, 2.5) {input};
    \end{tikzpicture}
  \end{figure}
\end{frame}

\begin{frame}{Pipe}
  Per prendere l'output di un comando e riderizionarlo in input verso un altro
  comando si usa la \textbf{pipe} \texttt{|}\bigskip

  Per esempio se vogliamo vedere tutti i file presenti in \texttt{/bin} il 
  nostro terminale si riempie di scritte.\bigskip

  Possiamo visualizzare il lungo output con il comando \texttt{less}: 
  \texttt{ls /bin | less}
\end{frame}

\begin{frame}[fragile]{Pipe - grep}
  Come abbiamo visto la lista di file presenti in \texttt{/bin} è molto lunga.
  Se volessimo trovarne uno specifico? \bigskip

  Nonostante esiste un comando apposito per cercare file, con le conoscenze che 
  abbiamo al momento possiamo costruire una soluzione alternativa:\bigskip\pause

  \begin{figure}
    \begin{lstlisting}
[matti@pirandello prove]$ ls /bin | grep "firefox" 
firefox
[matti@pirandello prove]$
    \end{lstlisting}
  \end{figure}
  
  dove al posto di \textit{firefox} può andarci una qualsiasi stringa. \bigskip
\end{frame}

\begin{frame}{Pipe - esempi}
  Di seguito una lista di esempi di utilizzo della \texttt{pipe}:
  \begin{itemize}
    \item \texttt{cat file1 file2 | grep "word"} cerca una stringa in più file
    \item \texttt{ls /bin | wc -l} conta quanti programmi sono presenti in 
      \texttt{/bin}
    \item \texttt{ls /bin | grep "zip" | wc -l} conta quanti programmi hanno
      la stringa "\textit{zip}" al loro interno nella cartella \texttt{/bin}
    \item \texttt{grep "castoro" animali | wc -l} conta le occorrenze di 
      \texttt{castoro} trovate nel file \texttt{animali}
    \item \texttt{grep "the" book | less} mostra le occorrenze di \texttt{the} 
      trovate in \texttt{book} attraverso il lettore \texttt{less}
  \end{itemize}
\end{frame}

\begin{frame}{Ridirezione su file}
  Visto che è possibile mandare l'output di un comando nell'input di un altro 
  comando, come possiamo salvare l'output di un comando su un file?

  Esiste l'\textbf{operatore di ridirezione} \texttt{>}\bigskip

  Sintassi: \texttt{comando > file}\bigskip

  \textbf{ATTENZIONE}: Alla \textit{shell} non interessa se il file esiste già, quindi se 
  esiste lo SOVRASCRIVE COMPLETAMENTE.
\end{frame}

\begin{frame}[fragile]{Esempi ridirezione distruttiva su file}
  \begin{figure}
    \begin{lstlisting}
[matti@pirandello prove]$ ls
capybara  castoro  criceto
[matti@pirandello prove]$ ls > lista
[matti@pirandello prove]$ ls
capybara  castoro  criceto lista
[matti@pirandello prove]$ cat lista
capybara  castoro  criceto lista 
// !contiene anche lista!
[matti@pirandello prove]$ echo "Hello World" > lista
[matti@pirandello prove]$ cat lista
// cosa contiene ora lista?
    \end{lstlisting}
  \end{figure}
\end{frame}

\begin{frame}{Ridirezione su file non distruttiva}
  Esiste anche un operatore per indirizzare su file l'output di un comando senza 
  sovrascrivere il contenuto del file, ma "appendendo" alla fine del file il
  contenuto scritto.\bigskip

  Sintassi: \texttt{comando >> file}\bigskip

  Si usa nello stesso modo dell'operatore classico\bigskip
\end{frame}

\begin{frame}[fragile]{Esempi ridirezione non distruttiva su file}
  \begin{figure}
    \begin{lstlisting}
[matti@pirandello prove]$ ls
capybara  castoro  criceto lista
[matti@pirandello prove]$ cat lista
Hello World
[matti@pirandello prove]$ ls | wc -l >> lista
[matti@pirandello prove]$ cat lista
// cosa contiene ora lista?
    \end{lstlisting}
  \end{figure}
\end{frame}
