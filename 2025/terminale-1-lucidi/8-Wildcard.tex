\section{Wildcard}
\begin{frame}{Wildcard}
  Nella \textit{shell} l'asterisco \texttt{*} fa da "segnaposto" per una 
  qualsiasi altra sequenza di caratteri.\bigskip

  Esempio: \texttt{torr*} si espande in: \texttt{torr}, \texttt{torra}, 
  \texttt{torrb}, $\cdots$, \texttt{torraa}, \texttt{torrab}, \texttt{torrac}. 
  Valgono ovviamente anche i numeri e altri caratteri oltre alle lettere.\bigskip

  Questo permette di indicare più file con parti comuni nel nome. Si possono
  combinare anche più asterischi: \texttt{c*a*} fa match con tutte le
  parole che iniziano per \texttt{c} e hanno almeno una \texttt{a} nel 
  nome. \texttt{*pila*} fa match con tutte le parole che hanno \texttt{pila} nel 
  nome.\bigskip
\end{frame}

\begin{frame}[fragile]{Wildcard esempi}
  \begin{figure}
    \begin{lstlisting}[basicstyle=\footnotesize]
[matti@pirandello castoro]$ cd ..
[matti@pirandello prove]$ ls
capybara  castoro  criceto
    \end{lstlisting}
  \end{figure}

  Proviamo a stampare solo le cartelle che iniziano con \texttt{ca}: 
  
  \begin{figure}
    \begin{lstlisting}[basicstyle=\footnotesize]
[matti@pirandello prove]$ ls -d ca*
capybara  castoro
    \end{lstlisting}
  \end{figure}

  La flag \texttt{-d} del comando \texttt{ls} permette di elencare solo le
  directory \bigskip

  Perché dobbiamo metterla se vogliamo elencare solo le cartelle? Cosa
  stamperebbe altrimenti?

\end{frame}