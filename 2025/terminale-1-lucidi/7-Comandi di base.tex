\section{Complementi di comandi base}

\begin{frame}{Complementi di comandi base}
  - Per pulire il terminale esiste il comando \texttt{clear}\bigskip

  - Per resettare il terminale esiste il comando \texttt{reset}. Sarà molto utile
  quando lavorerete al progetto di programmazione e romperete tutto con 
  la libreria grafica.\bigskip

  - Per vedere i vecchi comandi eseguiti esite il comando \texttt{history}.
    Digitando solamente history vedrete gli ultimi comandi usati, idendificati 
    dal numero del comando. \bigskip

  - Per riprendere un comando eseguito di recente basta utilizzare la freccetta
  verso l'alto. \bigskip
\end{frame}

\begin{frame}{Copia, incolla e interruzione}
  Nel terminale non funziona il classico copia e incolla da tastiera eseguito
  con \texttt{ctrl + c} e \texttt{ctrl + v}. Queste combinazioni di tasti hanno 
  il loro scopo e non sono fatti per copiare.\bigskip

  Per copiare e incollare dovete usare \texttt{shift + ctrl + c} e 
  \texttt{shift + ctrl + v}.\bigskip

  \texttt{ctrl + c} serve per interrompere un processo in esecuzione.
\end{frame}