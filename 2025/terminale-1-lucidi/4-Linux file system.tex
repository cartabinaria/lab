\section{Linux file system}

\subsection{Cos'è un file system}
\begin{frame}{Cos'è un file system}
  Il termine file system assume vari significati
  \begin{itemize}
    \item L'insieme dei file e delle directory che sono accessibili ad una macchina Linux
    \item L'organizzazione logica utilizzata da un s.o. per gestire un insieme di file in
memoria secondaria
    \item Il termine viene anche utilizzato per indicare una singola unità di memoria
secondaria
  \end{itemize}

  I file-system di Linux sono gerarchici, cioè organizzati ad albero. La radice
  è \texttt{/} e directory, sotto-directory e file sono i nodi dell'albero. \bigskip

  I \textit{path} sono unici, non possono esserci quindi file diversi con lo 
  stesso \textit{path}. \bigskip

  Ne consegue che se due file hanno lo stesso \textit{path} sono lo stesso file.
  \bigskip
\end{frame}

\begin{frame}[fragile]{Cartelle di sistema}
  \begin{figure}
    \begin{lstlisting}[basicstyle=\footnotesize]
[mattia@pirandello ~]$ cd /
[mattia@pirandello /]$ ls
bin  dev  home  mnt  opt  run   tmp  var   boot  root
[mattia@pirandello /]$ cd
[mattia@pirandello ~]$ cd Documents/prove/
    \end{lstlisting}
    \caption{Lista dei file presenti in \texttt{/}}
  \end{figure}
  \begin{itemize}
    \item \texttt{/bin}: contiene programmi necessari al sistema per funzionare
    \item \texttt{/boot}: contiene il kernel e altri file necessari al sistema
      per partire.
    \item \texttt{/etc}: contiene tutti i file di configurazione del sistema.
    \item \texttt{/home}: contiene le cartelle riservate agli utenti
    \item \texttt{/tmp}: contiene file temporanei che vengono cancellati ad ogni
      spegnimento
  \end{itemize}
\end{frame}

\subsection{\textit{Path} relativi e assoluti}
\begin{frame}{\textit{Path} relativi e assoluti}
  Per identificare un file è possibile usar due tipi di \textit{path}: assoluto 
  e relativo:
  \begin{enumerate}
    \item Un \textit{path} assoluto è un percorso ad un file che inizia da 
      \texttt{/} e termina con il nome di quel file. prende quindi le seguenti
      forme:
      \begin{itemize}
        \item \texttt{/home/mattia/slides.tex}
        \item \texttt{/usr/bin/firefox}
        \item \texttt{/tmp}
      \end{itemize}
    \item Un \textit{path} relativo è il percorso necessario per raggiungere un
      file rispetto alla \textbf{cartella corrente}. prende quindi le seguenti
      forme:
      \begin{itemize}
        \item \texttt{slides.tex}
        \item \texttt{../}
        \item \texttt{immagini/greg.png}
      \end{itemize}
  \end{enumerate}
\end{frame}
