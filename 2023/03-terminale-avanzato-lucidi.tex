\documentclass{beamer}
\usepackage[italian]{babel}
\usepackage[utf8]{inputenc}
\usepackage{graphicx}
\usepackage{subfig}
\usepackage{hyperref}
\usepackage{tikz}
\usetikzlibrary{shapes, calc, positioning}
\usepackage{tabularx}
\usepackage{listings}
\usepackage{color}

\definecolor{mygreen}{rgb}{0,0.6,0}
\definecolor{mygray}{rgb}{0.5,0.5,0.5}
\definecolor{mymauve}{rgb}{0.58,0,0.82}

\lstset{ 
  backgroundcolor=\color{black!20!white},   % choose the background color
  basicstyle=\ttfamily,        % the size of the fonts that are used for the code
  captionpos=b,                    % sets the caption-position to bottom
  commentstyle=\color{mygreen},    % comment style
  frame=single,	                   % adds a frame around the code
  %keepspaces=true,                 % useful for keeping indentation 
  language=bash,
  rulecolor=\color{black},         % if not set, the frame-color may be changed on line-breaks within not-black text (e.g. comments (green here))
  showspaces=false,                % show spaces everywhere adding particular underscores; it overrides 'showstringspaces'
  showtabs=false,                  % show tabs within strings adding particular underscores
  tabsize=2,	                   % sets default tabsize to 2 spaces
    deletekeywords={cd},            % if you want to delete keywords from the given language
}

\title{Terminale avanzato}
\author{Samuele Musiani, Alice Benatti}
\institute{Università di Bologna, corso di Laurea in Informatica}
\date{23 novembre 2023}
\logo{\includegraphics[width=0.05\textwidth]{assets/by-nc-sa-4-0.png}}

\AtBeginSection[]{
  \begin{frame}
    % \setcounter{tocdepth}{10}
  \vfill
  \centering
  \begin{beamercolorbox}[sep=8pt,center,shadow=true,rounded=true]{title}
    \usebeamerfont{title}\insertsectionhead\par%
  \end{beamercolorbox}
  % \tableofcontents[currentsubsection]
  \vfill
\end{frame}
}

\begin{document}

\begin{frame} 
  \titlepage
\end{frame}

\section{Un mare di shell}

\subsection{Cos'è una shell}
\begin{frame}[fragile]{Cos'è una shell}
  Una \texttt{shell} è un programma che permette di parlare con il sistema 
  operativo attraverso dei comandi da tastiera.
  \pause

  \begin{figure}
        \begin{lstlisting}
[samu@leibniz ~]$
    \end{lstlisting}
    \caption{Esempio di un \textit{shell prompt}}
  \end{figure}
\end{frame}

\subsection{Cos'è una shell}
\begin{frame}[fragile]{Cos'è una shell}
  Esistono varie shell:
  \begin{itemize}
    \item sh
    \item bash
    \item zsh
    \item fish
  \end{itemize}
  Tutte servono per interagire con il sistema operativo, però sono tutte 
  leggermente diverse sotto alcuni aspetti.\medskip

  Di solito la predefinita nei sistemi Linux è bash.
\end{frame}

\section{I primi passi in un terminale}
\begin{frame}{Aprire un terminale}
  Per utilizzare una \texttt{shell} è necessario disporre di un emulatore di 
  terminale.\bigskip

  Un emulatore di terminale è un'applicazione che permette di interagire con
  la \texttt{shell}.\medskip

  Esisto vari tipi di emulatori che offrono configurazioni e opzioni diverse:
  \begin{itemize}
    \item Alacritty
    \item Foot
    \item Kitty
    \item Konsole
    \item ...
  \end{itemize}
\end{frame}

\section{Pipe e ridirezione: il potere della shell}
\begin{frame}{Il potere della shell}
  Di solito si utilizzano comandi singoli. Nonostante molti di essi siano utili 
  anche da soli, sono sicuramente più utili usati in combinazione con altri 
  comandi.\bigskip

  Come si fa però a combinare più comandi? \pause
  Ricordiamoci che:
  \begin{itemize}
    \item <2-> Un comando restituisce sempre\only<2-4>{\footnote{In realtà non 
      tutti i comandi restituiscono per forza qualcosa, es. \texttt{cd}
      \vspace{3mm}}} qualcosa sullo standard output
    \item <3-> Lo standard output è considerato come un file (in memoria) dal 
      sistema 
    \item <4-> La maggior parte dei comandi visti fino ad adesso hanno la 
      possibilità di prendere in input lo standard output invece che un file 
      classico
  \end{itemize}
\end{frame}

\begin{frame}{Il potere della shell}
  \begin{figure}
    \begin{tikzpicture}
      \draw (-1.5,3) rectangle ++(3, 1);
      \draw (-1.5,-3) rectangle ++(3, 1);
      \draw [->] (0,3) -- (0, 1);
      \draw [->] (0,-0.9) -- (0, -2);
      \node [cylinder, shape border rotate=90, draw,minimum height=2cm,minimum width=1.5cm] {};

      \node {buffer};
      \node at (0,3.5) {command 1};
      \node at (0,-2.5) {command 2};
      \node at (1, 2) {output};
      \node at (1,-1.5) {input};
    \end{tikzpicture}
  \end{figure}
\end{frame}

\begin{frame}{Pipe}
  Per prendere l'output di un comando e riderizionarlo in input verso un altro
  comando si usa la \textbf{pipe} \texttt{|}\bigskip

  Per esempio se vogliamo vedere tutti i file presenti in \texttt{/bin} il 
  nostro terminale si riempie di scritte.\bigskip

  Possiamo visualizzare il lungo output con il comando \texttt{less}:\smallskip
  \texttt{ls /bin | less}
\end{frame}

\begin{frame}{Pipe - grep}
  Come abbiamo visto la lista di file presenti in \texttt{/bin} è molto lunga.
  Se volessimo trovarne uno specifico?\pause \bigskip

  Nonostante esiste un comando apposito per cercare file, possiamo fare:
  \texttt{ls /bin | grep "firefox"} dove al posto di \textit{firefox} può 
  andarci una qualsiasi stringa.
\end{frame}

\begin{frame}{Pipe - esempi}
  Di seguito una lista di esempi di utilizzo della \texttt{pipe}:
  \begin{itemize}
    \item \texttt{cat file1 file2 | grep "word"} cerca una stringa in più file
    \item \texttt{ls /bin | wc -l} conta quanti programmi sono presenti in 
      \texttt{/bin}
    \item \texttt{ls /bin | grep "zip" | wc -l} conta quanti programmi hanno
      la stringa "\textit{zip}" al loro interno nella cartella \texttt{/bin}
    \item \texttt{grep "castoro" animali | wc -l} conta le occorrenze di 
      \texttt{castoro} trovate nel file \texttt{animali}
    \item \texttt{grep "the" book | less} mostra le occorrenze di \texttt{the} 
      trovate in \texttt{book} attraverso il lettore \texttt{less}
  \end{itemize}
\end{frame}

\begin{frame}{Ridirezione su file}
  Come abbiamo visto è possibile mandare l'output di un comando nell'input di
  un altro comando.\bigskip

  Se volessimo salvare l'output di un comando su un file?\pause

  Esiste l'\textbf{operatore di ridirezione} \texttt{>}\bigskip

  Al posto di indirizzare l'output in un comando, scrive direttamente su un
  file.\bigskip

  Sintassi: \texttt{comando > file}\bigskip

  ATTENZIONE: Alla \textit{shell} non interessa se il file esiste già, quindi se 
  esiste lo SOVRASCRIVE COMPLETAMENTE\only<2>{\footnote{In reltà in shell più 
  moderne tipo zsh esce un prompt se il file esiste già\vspace{3mm}}}.
\end{frame}

\begin{frame}{Ridirezione su file non distruttiva}
  Esiste anche un operatore per indirizzare su file l'output di un comando senza 
  sovrascrivere il contenuto del file, ma "appendendo" alla fine del file il
  contenuto scritto.\bigskip

  Sintassi: \texttt{comando >> file}\bigskip

  Si usa nello stesso modo dell'operatore classico\bigskip
\end{frame}

\begin{frame}[fragile]{Ridirezione dello standard error}
  Di predefinito l'operatore di redirezione, come la pipe, reindirizza solo
  lo standard output. Questo significa che se un comando restituisce un errore
  questo non verrà riportato nel file indicato:

  \begin{figure}
    \begin{lstlisting}
[samu@leibniz ~]$ ls /asdkjc > /tmp/castoro
ls: cannot access '/asdkjc': No such file or 
directory
[samu@leibniz ~]$
    \end{lstlisting}
    \caption{Esempio di stderr non ridirezionato}
  \end{figure}

  Per ridirezionare anche lo standard error dentro il file è necessario
  specificarlo alla shell.
\end{frame}

\begin{frame}[fragile]{Ridirezione dello standard error}
  Lo standard error in linux ha come \textit{file descriptor} il \textbf{2}. 
  Non importa sapere cosa sia un file descriptor, basta ricordarsi che
  \textbf{stdout = 1} e \textbf{stderr = 2}. \medskip

  Per ridirezionare lo stderr sullo stdout si usa la seguente sintassi:
  \textbf{2>\&1} \medskip

  Redirezionando lo standard error verso lo standard output è possibile scrivere
  entrambi su file:
  \begin{figure}
    \begin{lstlisting}
[samu@leibniz ~]$ ls /asdkjc > /tmp/test 2>&1 
[samu@leibniz ~]$ cat /tmp/test
ls: cannot access '/asdkjc': No such file or 
directory
    \end{lstlisting}
    \caption{Esempio di stderr ridirezionato}
  \end{figure}
\end{frame}

\begin{frame}[fragile]{Ridirezione dello standard error}
  Da notare che questo funziona anche per la pipe:
  \begin{figure}
    \begin{lstlisting}
[samu@leibniz ~]$ ls /asdkjc | head -c 10
ls: cannot access '/asdkjc': No such file or 
directory
[samu@leibniz ~]$ ls /asdkjc 2>&1 | head -c 10
ls: cannot
    \end{lstlisting}
    \caption{Esempio di stderr ridirezionato}
  \end{figure}
\end{frame}

\begin{frame}{Filtri}
  Quando si combinano assieme più comandi spesso i seguenti tornano utili:
  \begin{itemize}
    \item<1-> sort: Restituisce in stdout il testo preso in stdin
    \item<2-> uniq: Se ci sono più linee consecutive uguali ne lascia solo una
    \item<3-> grep: Cerca il testo passato come argomento
    \item<4-> wc: Conta linee, parole e carateri passati in input
    \item<5-> head: Manda in stdout solo la prima parte del testo
    \item<6-> tail: Manda in stdout solo l'ultima parte del testo
  \end{itemize}
\end{frame}

\begin{frame}[fragile]{/dev/null}
  Come si fa a ridirezionare soltanto lo stderr su file e "buttare via" lo
  stdout? \medskip
  \pause

  Esiste in Linux un "file" che indipendentemente da cosa gli passate lo butta 
  via. Questo "file" è \textbf{/dev/null}.\medskip

  \begin{figure}
    \begin{lstlisting}
[samu@leibniz ~]$ ls /asdkjc / 2> /tmp/pippo 
1> /dev/null
[samu@leibniz ~]$ cat /tmp/pippo
ls: cannot access '/asdkjc': No such file or 
directory
    \end{lstlisting}
    \caption{Esempio di stderr ridirezionato}
  \end{figure}

Ricordiamoci che stdout = 1 e stderr = 2.
\end{frame}

\section{\Large echo, \small echo, \tiny echo, echo...}
\begin{frame}[fragile]{Echo, \small echo, \tiny echo...}
  Il comando \texttt{echo}, come suggerisce il suo nome, serve per stampare
  quello che gli viene passato come argomento:

  \begin{figure}
    \begin{lstlisting}
[samu@leibniz ~]$ echo "ciao"
ciao
[samu@leibniz ~]$ echo "Questa è una frase"
Questa è una frase
    \end{lstlisting}
    \caption{Esempio di echo}
  \end{figure}
  A prima vista può sembra abbastanza inutile, in realtà ha tantissime
  applicazioni.
\end{frame}

\begin{frame}[fragile]{Seeing the world as the shell sees it}
  \begin{figure}
    \begin{lstlisting}
[samu@leibniz ~]$ echo *
Documents Downloads go Pictures VirtualBox
[samu@leibniz ~]$ echo /usr/*/share
/user/kerberos/share /user/local/share
    \end{lstlisting}
    \caption{Esempio di echo con una wildcard}
  \end{figure}

  Come possiamo vedere echo non stampa l'asterisco come potremmo aspettarci. 
  Lo tratta invece come la shell: ovvero come una wildcard. \medskip \pause

  In realtà è proprio la shell a sostituire la wildcard con le entries opportune
  prima di eseguire il comando \texttt{echo}.
\end{frame}

\begin{frame}[fragile]{Seeing the world as the shell sees it}
  Esiste una particolare sintassi per indicare un range in bash:
  \begin{figure}
    \begin{lstlisting}
[samu@leibniz ~]$ echo {1..11}
1 2 3 4 5 6 7 8 9 10 11
[samu@leibniz ~]$ echo {c..t}
c d e f g h i j k l m n o p q r s t
    \end{lstlisting}
    \caption{Esempio di echo range}
  \end{figure}

  La shell, prima di esegure l'echo, sostituisce le parentesi graffe con il 
  range specificato.\medskip \pause

  Non funziona solo con echo in quanto è la shell stessa a fare la sostituzione.
  Funziona molto bene con \texttt{mkdir}: \smallskip 

  \texttt{mkdir \{2020..2021\}-\{01..12\}-\{01..31\}}
\end{frame}

\section{Sostituzione di comandi}
\begin{frame}[fragile]{La pipe non riesce a fare tutto}
  Con la pipe possiamo combinare più comandi assieme, però non riusciamo a fare
  ancora tutto. \medskip

  Se volessimo vedere il tipo di tutti i file che hanno la parola zip nel nome
  dentro la cartella \texttt{/bin} come facciamo? \medskip \pause

  \begin{figure}
    \begin{lstlisting}
$ ls -d /bin/* | grep zip | file
    \end{lstlisting}
    \caption{Esempio SBAGLIATO. Il prompt è stato tralasciato per questioni di 
    spazio. Al suo posto è utilizzato soltanto il dollaro}
  \end{figure}
  
  Questo comando restituisce un errore perché il comando \texttt{file} non 
  riesce ad interpretare lo standard input tramite la pipe. \pause

  Il comando file vuole gli input come argomenti: \texttt{file ...}
\end{frame}

\begin{frame}[fragile]{Sostituzione di comandi}
  La shell ammette una sintassi per la sostituzioni di comandi all'interno di
  un'espressione. \medskip

  Sintassi: \texttt{\$(comando)} \medskip

  La shell sostituirà l'intera espressione al risulto del comando:
  \begin{figure}
    \begin{lstlisting}
[samu@leibniz ~]$ echo $(ls)
Documents Downloads go Pictures VirtualBox
[samu@leibniz ~]$ ls
Documents Downloads go Pictures VirtualBox
    \end{lstlisting}
    \caption{Esempio di sostituzione del comando \texttt{ls}}
  \end{figure}
\end{frame}

\begin{frame}[fragile]{Sostituzione di comandi}
  Ritornando a quello che volevamo fare prima: \textit{Se volessimo vedere il 
  tipo di tutti i file che hanno la parola zip nel nome dentro la cartella 
  \texttt{/bin} come facciamo?} \pause Con la sostituzione di comandi! \medskip

  \begin{figure}
    \begin{lstlisting}
$ file $(ls -d /bin/* | grep zip)
    \end{lstlisting}
    \caption{Esempio di più comandi. Il prompt è stato tralasciato. Al suo posto
    è utilizzato un dollaro (il primo).}
  \end{figure}
\end{frame}

\begin{frame}[fragile]{Matematica base}
  Si possono fare anche operazioni matematiche con questo tipo di sostituzione:
  \begin{figure}
    \begin{lstlisting}
[samu@leibniz ~]$ echo $((1 + 1))
2
[samu@leibniz ~]$ echo $((9 * 5))
45
[samu@leibniz ~]$ echo 3GB are $((3 * 1024))MB
3GB are 3072MB
    \end{lstlisting}
    \caption{Esempio di più comandi. Il prompt è stato tralasciato. Al suo posto
    è utilizzato un dollaro (il primo).}
  \end{figure}
\end{frame}

\section{Variabili}
\begin{frame}[fragile]{Variabili}
  È possibile salvare dei valori in delle \textbf{variabili}. \medskip

  Alcuni programmi sono configurabili attraverso le variabili di ambiente. 
  \medskip

  Esistono delle variabili già esistenti nella shell che possono essere viste
  con il comando \texttt{echo}. \medskip

  \begin{figure}
    \begin{lstlisting}
[samu@leibniz ~]$ echo $USER
samu
[samu@leibniz ~]$ echo $SHELL
/bin/bash
    \end{lstlisting}
    \caption{Esempio variabile}
  \end{figure}

  Le variabili saranno più utili negli script che nella shell direttamente. 
  \medskip
\end{frame}

\begin{frame}[fragile]{Assegnamento delle variabili}
  Per assegnare un valore ad una variabile si può usare la seguente sintassi:
  \texttt{varname=content}.

  \begin{figure}
    \begin{lstlisting}
[samu@leibniz ~]$ pippo=cane
[samu@leibniz ~]$ echo $pippo
cane
    \end{lstlisting}
    \caption{Esempio di assegnamento variabile}
  \end{figure}

  Una variabile è vuota se quando si printa con \texttt{echo} non ha nulla al suo
  interno:
  \begin{figure}
    \begin{lstlisting}
[samu@leibniz ~]$ echo $papera

[samu@leibniz ~]$ 
    \end{lstlisting}
    \caption{Esempio di variabile vuota}
  \end{figure}
\end{frame}

\begin{frame}[fragile]{Uso di variabili}
  Si possono utilizzare le variabili nei comandi:

  \begin{figure}
    \begin{lstlisting}
[samu@leibniz ~]$ src=/home/samu/Immagini
[samu@leibniz ~]$ dest=/tmp/immagini
[samu@leibniz ~]$ cp -r $src $dest
    \end{lstlisting}
    \caption{Esempio di copia utilizzando variabili}
  \end{figure}

  \begin{figure}
    \begin{lstlisting}
[samu@leibniz ~]$ programs=$(ls /bin)
[samu@leibniz ~]$ echo $programs | less
    \end{lstlisting}
    \caption{Tutti i programmi in \texttt{/bin} sono salvati nella variabile 
    \textit{programs}}
  \end{figure}
\end{frame}

\begin{frame}[fragile]{La variabile path}
  Una variabile spesso usata è la variabile di \textbf{PATH}. \medskip

  Quando digitato un comando la shell andrà a cercare se esiste un eseguibile
  con lo stesso nome del comando nelle directory indicate dalla variabile
  \texttt{PATH}. \medskip

  \begin{figure}
    \begin{lstlisting}
[samu@leibniz ~]$ echo $PATH
/usr/local/bin:/usr/local/sbin:/usr/bin:/usr/l
ib/jvm/default/bin:/usr/bin/site_perl
[samu@leibniz ~]$
    \end{lstlisting}
    \caption{Percorsi nella variabile \texttt{PATH}}
  \end{figure}
\end{frame}

\begin{frame}[fragile]{Export}
  Di predefinito le variabili non sono propagate anche sui processi figli della
  shell. \medskip

  È quindi spesso necessario esportarle per renderle visibili. \medskip
  \begin{figure}
    \begin{lstlisting}
[samu@leibniz ~]$ pippo=cane
[samu@leibniz ~]$ echo $pippo
cane
[samu@leibniz ~]$ bash -c 'echo $pippo'

[samu@leibniz ~]$ export pippo=cane
[samu@leibniz ~]$ bash -c 'echo $pippo'
cane
    \end{lstlisting}
    \caption{Percorsi nella variabile \texttt{PATH}}
  \end{figure}
\end{frame}

\section{Processi}

\begin{frame}{htop}
  Linux è sistema che permette l'esecuzione di più processi simultaneamente. 
  \medskip

  Per vedere i processi in esecuzione (e altre informazioni) è molto comodo il 
  comando \texttt{top} o \texttt{htop} (la versione nuova). \medskip
  
  Se si vuole invece aver uno snapshot dei processi in esecuzione si può 
  utilizzare il comando \texttt{ps}. \medskip

  Per vedere tutti i processi su \texttt{ps} la flag è \texttt{-e}.
\end{frame}

\begin{frame}[fragile]{Pid e kill}
  Per identificare in modo univoco un processo si utilizza un intero positivo
  chiamato \textbf{pid}: Process IDentifier. \medskip \pause

  Se per caso volessimo terminare un processo lo si può fare attraverso il 
  comando \texttt{kill} passando il pid del processo da uccidere. \medskip

  Il pid può essere ottenuto attraverso il comando \texttt{ps}:
  \begin{figure}
    \begin{lstlisting}
[samu@leibniz ~]$ ps
    PID TTY          TIME CMD
   8239 pts/3    00:00:03 zsh
  20813 pts/3    00:00:00 ps
[samu@leibniz ~]$ kill 8239
    \end{lstlisting}
    \caption{Esempio di \texttt{kill}}
  \end{figure}
\end{frame}

\section{Bash scripting}

\begin{frame}[fragile]{Bash scripting}
  Spesso si vogliono automatizzare dei comandi della shell, oppure eseguirne più
  in "una volta sola". \medskip \pause

  La shell (e in particolare bash) permette di scrivere degli script. \medskip

  Uno script è un file di test con un elenco di comandi che vengono eseguiti in
  sequenza da bash\only<2>{\footnote{In realtà dall'interprete che specificate}}
  .\medskip

  Per specificare che interprete da usare è necessario indicare nella prima riga
  del file il percorso assoluto all'eseguibile della shell: \smallskip

  \begin{figure}
    \begin{lstlisting}
#!/bin/bash
    \end{lstlisting}
    \caption{Riga per indicare bash come interprete}
  \end{figure}
\end{frame}

\begin{frame}[fragile]{Esempio di script}
  \begin{figure}
    \begin{lstlisting}
#!/bin/bash
echo "Reading programs"
PROG=$(ls -d /bin/*)
NUM=$(echo "$PROG" | grep "ca" | wc -l)
echo "Numero di programmi con 'ca' è $NUM"
    \end{lstlisting}
    \caption{Esempio di bash script}
  \end{figure}

  Questo script fa le seguenti operazioni:
  \begin{enumerate}
    \item Stampa sullo standar output "\textit{Reading programs}"
    \item Salva tutti i programmi presenti in \texttt{/bin} nella variabile 
      \texttt{PROG}.
    \item Salva nella variabile \texttt{NUM} il numero di tutte le stringhe 
      presenti nella variabile \texttt{PROG} che hanno al loro interno "ca".
    \item Stampa il valore presente nella variabile \texttt{NUM}.
  \end{enumerate}
\end{frame}

\begin{frame}[fragile]{Eseguire uno script}
  Per eseguire uno script bash è importante avere un file (lo script) con i 
  comandi al suo interno. \medskip

  La convenzione è che gli script bash si salvano con estenzione \textbf{.sh}
  \medskip

  Inoltre bisogna renderlo eseguibile aggiungendogli il permesso: \smallskip

  \texttt{chmod +x script.sh} \medskip

  Per eseguilo basta quindi indicare il suo percorso assoluto o relativo:
  \begin{figure}
    \begin{lstlisting}
[samu@leibniz scripts]$ ls -al
-rw-r--r-- 1 samu samu Nov 20 17:17 script.sh
[samu@leibniz ~]$ chmod +x script.sh
-rwxr-xr-x 1 samu samu Nov 20 17:17 script.sh
[samu@leibniz ~]$ ./script.sh
    \end{lstlisting}
    \caption{Esecuzione di uno script bash}
  \end{figure}
\end{frame}

\begin{frame}[fragile]{if, for, while}
  Non abbiamo il tempo di trattarli ma in bash esistono anche \textbf{if}, 
  \textbf{for} e \textbf{while}:
  \begin{figure}
    \begin{lstlisting}
#!/bin/bash
DEST="/mnt/usb/imgs"
IMGS=$(find ~/Images -type f -name "*.jpg")
a=1
for i in $IMGS; do
  echo "Coping $i to $DEST/$a.jpg"
  cp "$i" "$DEST/$a.jpg"
    a=$(($a+1))
done
    \end{lstlisting}
    \caption{Esempio di bash script}
  \end{figure}
\end{frame}

\section{Una veloce introduzione a vim}
\begin{frame}[fragile]{Un po' di contesto}
  Su Linux esistono due principali editor di testo: nano e vim. \smallskip

  Nano è un editor semplice che ha i comandi scritti a schermo per evitare di
  scordarseli. \smallskip \pause

  La filosofia di vim invece è diversa. Facciamo qualche osservazione: \pause
  \begin{itemize}[<+->]
    \item Quando si programma la maggior parte del tempo è passato a 
      \textit{modificare} il codice, non a scriverlo

    \item Modificare il codice include molto altro oltre a scrivere: eliminare,
      sostituire, riordinare, duplicare, formattare, ecc.

    \item Ha senso facilitare tutta la parte di modifica del codice, più che 
      di scrittura effettiva

    \item Il mouse è una perdita di tempo quando si deve scrivere, se si può
      fare tutto da tastiera in modo efficiente è meglio

    \item Un'operazione usa il minor numero di tasti possibili.
  \end{itemize}
\end{frame}

\begin{frame}{Modalità}
  In un editor classico ci si trova sempre nella modalità di \textit{modifca},
  ovvero se si schiacciano delle lettere vengono inserite nel file che si sta 
  modificando. \medskip \pause 

  Per la maggir parte degli editor (compreso nano) questa è l'unica modalità 
  presente. \medskip \pause

  Vim introduce tre modalità principali:
  \begin{itemize}
    \item Normal mode
    \item Insert mode
    \item Visual mode
  \end{itemize}
\end{frame}

\begin{frame}{Modalità}
  Vim introduce tre modalità principali:
  \begin{itemize}[<+->]
    \item \textbf{Normal mode}: Modalità predefinita di vim. La si attiva con 
      il tasto \texttt{ESC}. Permette di muoversi all'interno del documento in 
      modo rapido. Di eliminare, copiare, incollare e spostare parti di testo.

    \item \textbf{Insert mode}: Modalità classica per l'inserimento e la 
      scrittura di testo. È la stessa che si trova nella maggior parte degli 
      editor. Ci si accede principalmente usando il tasto 'i', ma anche 'I', 
      'a', 'A', ecc.

    \item \textbf{Visual mode}: Permette di selezionare parti di testo, sia 
      verticalmente che orizzontalmente. Ci si accede usanto il tato 'v'.
  \end{itemize}
\end{frame}

\begin{frame}{Normal mode}
  La \textbf{normal mode} è la predefinita e qulla su cui si dovrebbe sempre 
  stare se non si necessitano di altre modalità. Per entrarci usare il tasto
  \texttt{ESC}. \medskip

  Per spostarsi all'interno del testo in normal mode si usano le seguenti 
  lettere e NON le freccette:
  \begin{itemize}
    \item j: Per spostarsi verso il basso di una riga
    \item k: Per spostarsi verso l'alto di una riga
    \item h: Per spostarsi a sinistra di un carattere
    \item l: Per spostarsi a destra di un carattere
  \end{itemize}
  Il motivo per cui si utilizzano queste lettere e non le freccette è il fatto 
  che lo spostamento della mando per raggiungere le freccette è troppo grande
  e richiederebbe troppo tempo per la filosofia di vim.
\end{frame}

\begin{frame}{Normal mode}
  Altre combinazioni di tasti base in normal mode:
  \begin{itemize}
    \item w: si sposta all'inizio della parola successiva
    \item e: si sposta alla fine della parola successiva
    \item b: si sposta all'inizio della parola precedente
    \item gg: si sposta all'inizio del file
    \item G: si sposta alla fine del file
    \item dd: cancella la riga su cui si trova il cursore
    \item yy: copia la riga su cui si trova il cursore
    \item p: incolla il testo copiato nella riga successiva al cursore
    \item x: elimina il carattere sotto il cursore
    \item u: annulla l'ultima operazione fatta
    \item ctrl + r: ripete l'ultima operazione annullata (il contrario di 'u')
  \end{itemize}
\end{frame}

\begin{frame}{Insert mode}
  Si utilizza la \textbf{insert mode} quando si vuole scrivere del testo. Per
  entrarci dalla normal mode si possono usare i seguenti tasti:
  \begin{itemize}
    \item i: si inizia a digitare prima della lettera sotto il cursore
    \item a: si inizia a digitare dopo della lettera sotto il cursore
    \item I: si inizia a digitare all'inizio della riga
    \item A: si inizia a digitare alla fine della riga
    \item o: si inizia a digitare in una nuova linea sottostante
    \item O: si inizia a digitare in una nuova linea sovrastante
    \item cc: si cancella la riga su cui si trova il cursore e si inizia a 
      scrivere al suo posto
  \end{itemize}
\end{frame}

\begin{frame}{How to exit vim}
  Uscire da vim è una delle operazioni più complesse. \medskip \pause

  Molti dicono che l'unico modo sia riavviare il pc. \medskip \pause

  In realtà basta sapere come funziona! \medskip

  Per uscire da vim bisogna trovarsi in normal mode e digitare \textbf{:q} e 
  successivamente premere invio.\medskip

  Se il file è stato modificato dall'apertura il comando non funzionerà perché
  vim non sa se vuoi salvare il file o no. Per salvarlo (ed uscire) si usa 
  \textbf{:wq} e per non salvarlo (ed uscire) \textbf{:q!} \medskip

  Per salvare quindi un file senza uscire basta \textbf{:w}
\end{frame}

\section{More work to do}
\begin{frame}{More work to do}
  Sia bash che vim sono programmi immensi che non si finisce mai di conoscerli a
  fondo. Di seguito sono riportati alcuni comandi che potete approfondire: 
  \medskip
  \begin{itemize}
    \item awk, sed, tr
    \item diff
    \item ip e ping
    \item rsync
    \item killall
    \item dd
    \item du, df
  \end{itemize}
  ATTENZIONE: Mai usare un comando prima di sapere cosa fa. Alcuni di questi
  possono anche essere pericolosi se usati senza un minimo di criterio.
\end{frame}
\end{document}
