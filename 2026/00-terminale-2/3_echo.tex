\section{\Large echo, \small echo, \tiny echo, echo...}
\begin{frame}[fragile]{echo, \small echo, \tiny echo...}
  Il comando \texttt{echo}, come suggerisce il suo nome, serve per stampare
  quello che gli viene passato come argomento:

  \begin{figure}
    \begin{lstlisting}
[mattia@pirandello prove]$ echo "pippo"
pippo
[mattia@pirandello prove]$ echo "Questa è una frase"
Questa è una frase
[mattia@pirandello prove]$
\end{lstlisting}
    \caption{Esempio di echo}
  \end{figure}
  A prima vista può sembra abbastanza inutile, in realtà ha tantissime
  applicazioni.
\end{frame}

\begin{frame}[fragile]{Seeing the world as the shell sees it}
  \begin{figure}
    \begin{lstlisting}
[mattia@pirandello prove]$ echo *
1 capybara castoro criceto file file1 file2 mangusta
[mattia@pirandello prove]$ echo */lontra
capybara/lontra castoro/lontra
\end{lstlisting}
    \caption{Esempio di echo con wildcard}
  \end{figure}

  Come possiamo vedere \texttt{echo} non stampa * ma lo tratta come la shell:
  ovvero come una \textit{wildcard}. \medskip

  In realtà è proprio la shell a sostituire la wildcard con le entries opportune
  prima di eseguire il comando \texttt{echo}.
\end{frame}

\begin{frame}[fragile]{Seeing the world as the shell sees it}
  Esiste una particolare sintassi per indicare un range in bash:
  \begin{figure}
    \begin{lstlisting}
[mattia@pirandello prove]$ echo {1..11}
1 2 3 4 5 6 7 8 9 10 11
[mattia@pirandello prove]$ echo {c..t}
c d e f g h i j k l m n o p q r s t
\end{lstlisting}
    \caption{Esempio di echo range}
  \end{figure}

  La shell, prima di eseguire l'echo, sostituisce le parentesi graffe con il 
  range specificato.\medskip

  Non funziona solo con echo in quanto è la shell stessa a fare la sostituzione.
  Funziona molto bene con \texttt{mkdir}: \smallskip 

  \code{mkdir \{2020..2021\}-\{01..12\}-\{01..31\}}
\end{frame}


\begin{frame}[fragile]{Seeing the world as the shell sees it}
  E si usa come i metalinguaggi dei prof:
  \begin{figure}
    \begin{lstlisting}
[mattia@pirandello prove]$ echo {1,11}
1 11
\end{lstlisting}
    \caption{Esempio di due elementi di un range}
  \end{figure}
  
  \begin{figure}
    \begin{lstlisting}
[mattia@pirandello prove]$ echo A{dm,pp}Staff
AdmStaff AppStaff
\end{lstlisting}
    \caption{Esempio con stringhe}
  \end{figure}

\end{frame}
